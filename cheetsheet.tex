\documentclass[12pt]{article}
\usepackage[utf8]{inputenc}
\usepackage{graphicx}
%\usepackage{xcolor}
\usepackage[a4paper,left=2cm,right=2cm,top=2cm,bottom=4cm]{geometry}

\begin{document}

\section{Cheatsheet für AUT FV}
\subsection{Matrizen}
Für eine Matrix M gilt:
\[ M \in regul\ddot{a}r\] 
wenn $ det(M) \neq 0 $. \\ \\
\textbf{Bestimmung der Eigenwerte} \\ \\
Die Eigenwerte $ \lambda_{n} $ können mithilfe der Eigenwert-Eigenvektor-Gleichung bestimmt werden. \\
Eigenwert-Eigenvektor-Gleichung:
\[ det(A - \lambda E) = 0 \]
\[ \Lambda = VA \ V^{-1} \] 
Die einzelnen Spalten von V sind die Eigenvektoren von A. \\ \\
\textbf{Symmetrieeigenschaften}\\ \\
Eine reell quadratische Matrix besitzt folgende Möglichkeiten der Symmetrie:\\
\[ symemetrisch: A = A^{T}\]
\[ antisymmetrisch: A = -A^{T} \]
\[ orthogonal: wenn \ A \in regul\ddot{a}r \ und \ A^{T} = A^{-1} \ bzw.\  A^{T} \cdot A = E \]
\textbf{Invertieren einer Matrix} \\ \\
Eine beliebige $ n \times n $ - Matrix wird nach folgender Vorschrift invertiert: \\
\[ A^{-1} = \frac{1}{det(A)} \cdot adj(A) \]
\underline{Berechnung der Adjunkten:} \\ \\
Für die Adjunkte gilt folgende Beziehung:
\[ adj(A) = Cof(A)^{T} \]
Für die Ermittlung der Kofaktormatrix wird der Formalismus des Laplace'schen Entwicklungssatzes benötigt. \\
Laplace'scher Entwicklungssatz:
\[ \sum_{i=0}^{n}(-1)^{i+j} \cdot a_{ij} \cdot det(A_{ij}) \ \ Entwicklung \ nach \ der \ j-ten \ Spalte\]
oder
\[ \sum_{j=0}^{n}(-1)^{i+j} \cdot a_{ij} \cdot det(A_{ij}) \ \ Entwicklung \ nach \ der \ i-ten \ Zeile\]
\end{document}

%%%%%%%%%%%%%%%%%%%%%%%%%%%%%%%%%%%%%%
%TODO: Labor 1
%	- Matrizenrechenregeln
%	- Ruhelagen (AUT 2.5ff)
% 	- Regelungstechnische Übertragungsglieder (AUT 3.8ff) 
% 
%  (1,2,3[3.4] AUT) 
%  (3.1 AUTFV)
% 
%%%%%%%%%%%%%%%%%%%%%%%%%%%%%%%%%%%%%%
%TODO: Labor 2: 
%	- 3,6[6.4] AUT
% 	- 3 AUTFV 
%%%%%%%%%%%%%%%%%%%%%%%%%%%%%%%%%%%%%%
%TODO: Labor 3: 
% 	- 2,3,4 AUTFV 
%%%%%%%%%%%%%%%%%%%%%%%%%%%%%%%%%%%%%%
